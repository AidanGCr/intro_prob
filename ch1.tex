\documentclass[12pt]{article}   % 'article' class with 12pt font size
\usepackage[english]{babel}
\usepackage{amsthm}

\theoremstyle{definition}
\newtheorem{definition}{Definition}[section]

\theoremstyle{remark}
\newtheorem*{remark}{Remark}

% Begin document
\begin{document}

% Simple title
\begin{center}
    \Large{\textbf{Chapter 1: Probability and Counting}} \\
\end{center}

% Introduction or general content section
\section{}
This section provides an overview of the chapter contents.

% Example: Definitions
\section{Definitions}
\begin{definition}[Definition 1]
	Description of the definition.
\end{definition}

% Example: Theorems and proofs
\section{Theorems and Proofs}
\textbf{Theorem 1.1}: Statement of the theorem.

\textit{Proof}: Provide the proof here. For example:
\[
E = mc^2
\]
End of proof.

% Example: Notes and commentary
\section{Notes}
Additional remarks or observations about the chapter.

% Example: Exercises (optional)
\section{Exercises}
If you're taking notes on exercises:
\begin{itemize}
    \item \textbf{Exercise 1}: Description of the exercise.
    \item \textbf{Solution}: Provide solution if applicable.
\end{itemize}

% Add more sections as needed

\end{document}

